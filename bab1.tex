
\chapter{Selayang Pandang Teknologi}
Apakah Anda pernah belajar menguasai suatu bahasa pemrograman atau teknologi baru dengan menggunakan satu buku saja sebagai pengantar? Jika jawabannya ya, maka buku itu haruslah bisa berbicara secara bertahap, dimulai dengan memperlakukan Anda sebagai pemula, menengah dan kemudian mahir. Buku ini akan berusaha melakukan hal tersebut. Anda tidak harus mahir PHP/CodeIgniter/jQuery untuk dapat mengambil manfaat dari buku ini. Buku ini akan memberikan panduan yang jelas dan terarah untuk membuat Anda mencapai tingkat kemahiran yang dibutuhkan.

Apakah sebenarnya Web 2.0 itu? Ia adalah istilah yang ditujukan untuk semua website yang menyajikan interaktifitas dengan pengguna dengan lebih baik. Anda dapat menggunakan banyak bahasa pemrograman web untuk mengembangkan website berjenis Web 2.0, namun disini akan ditunjukkan serangkaian teknologi yang dipakai, yaitu :
\begin{enumerate}
\item PHP
  Merupakan bahasa pemrograman Web yang dipergunakan paling luas di dunia web
\item CodeIgniter
  Merupakan salah satu framework PHP yang dapat digunakan untuk mengembangkan berbagai aplikasi web PHP baik yang kompleks maupun yang ringan.
\item jQuery
  Merupakan pustaka JavaScript yang akan meningkatkan web Anda dari Web 1.0 ke Web 2.0.
\end{enumerate}

Pilihan teknologi yang dipakai bisa sangat bervariasi. PHP bs digantikan oleh Java, .NET atau Ruby. CodeIgniter bs digantikan oleh CakePHP, Symphony atau Zend Framework. Dan jQuery bisa digantikan oleh Dojo Framework, scriptaculous atau MooTools. 

Selanjutnya kita akan menggali lebih dalam mengenai kombinasi teknolog yang akan dipakai pada buku ini.
\section{PHP}
PHP merupakan bahasa pemrograman Web yang paling luas dipergunakan saat ini. TODO : Tampilkan chart. Dipergunakan oleh Facebook (DSB u marketing), dapat menjadi gambaran betapa PHP sangat dapat dihandalkan untuk mengembangkan berbagai jenis aplikasi web dalam skala apapun. PHP tersedia di hampir semua sistem operasi, baik Windows, Mac OS X, Linux dan berbagai varian unix lainnya. Dikombinasikan dengan MySQL, PHP terbukti dapat menjadi senjata ampuh yang menjadikan programmer yang menguasai keahlian pengembangan dengannya, dapat memberikan berbagai solusi komputasi yang dibutuhkan.
\section{MySQL}
MySQL merupakan basis data RDBMS yang paling luas dan paling banyak dipergunakan di muka bumi. Kombinasi antara PHP dan MySQL sudah menjadi kombinasi resmi yang menjadi standar bahasa pemrograman Web. RDBMS ini free, sehingga programmer yang mempergunakannya tidak usah pusing-pusing memikirkan tentang masalah lisensi penggunannya.
\section{CodeIgniter}
Anda bisa saja langsung mengembangkan aplikasi Web jenis apapun hanya dengan menggunakan PHP semata, tanpa bantuan framework yang banyak tersedia di dunia PHP. Namun, Anda kemungkinan sekali akan mengalami kesulitan untuk mengatasi berbagai permasalahan, yang justru sudah diatasi oleh framework-framework tersebut. CodeIgniter merupakan salah satu framework yang powerful, mudah dikuasai dan dipergunakan luas oleh pengembang situs PHP.
\section{jQuery}
Saat ini, tidak mungkin rasanya membayangkan suatu situs web dikembangkan tanpa menggunakan pustaka JavaScript yang handal. Peningkatan interaktivitas pada aplikasi Web (TODO: gambar fish eye) akan meningkatkan kenyamanan pengguna yang mampir ke situs Anda, dan tentunya ini akan meningkatkan kualitas web Anda. jQuery merupakan salah satu pustaka JavaScript yang dengannya Anda akan mudah mengembangkan berbagai jenis interaktivitas pada aplikasi web dengan mudah dan sangat berdaya guna

Namun pilihan yang dipaparkan pada buku ini merupakan pilihan yang bebas. Anda dapat menggantikan PHP dengan JSP misalnya. Atau MySQL dengan MSSql. Atau jika Anda telah memilih PHP sebagai bahasa pengembangannya, Anda tetap bisa menggunakan framework PHP yang lain, semisal Yii Framework. Pada bagian pustaka JavaScript Anda dapat menggunakan Dojo Toolkit, MooTool atau scriptaculous sebagai pustaka JavaScript yang dapat dipergunakan.

Setiap pilihan memiliki market sharenya masing-masing, dan Anda tetap akan mendapatkan banyak manfaat dengan menguasainya satu demi satu. Penulis buku ini justru dulunya lebih menyukai menggunakan DojoToolkit ketimbang jQuery. Namun, menambah skill penguasaan teknologi yang lain sangat menarik dan menantang serta bermanfaat untuk kemajuan profesi seorang programmer, maka dari itu penulis terus belajar berbagai teknologi-teknologi yang ada di pasaran dunia IT.
