\section{Selayang Pandang Teknologi}
Apakah Anda pernah belajar menguasai suatu bahasa pemrograman atau teknologi baru dengan menggunakan satu buku saja sebagai pengantar? Jika jawabannya ya, maka buku itu haruslah bisa berbicara secara bertahap, dimulai dengan memperlakukan Anda sebagai pemula, menengah dan kemudian mahir. Buku ini akan berusaha melakukan hal tersebut. Anda tidak harus mahir PHP/CodeIgniter/jQuery untuk dapat mengambil manfaat dari buku ini. Buku ini akan memberikan panduan yang jelas dan terarah untuk membuat Anda mencapai tingkat kemahiran yang dibutuhkan.

Apakah sebenarnya Web 2.0 itu? Ia adalah istilah yang ditujukan untuk semua website yang menyajikan interaktifitas dengan pengguna dengan lebih baik. Anda dapat menggunakan banyak bahasa pemrograman web untuk mengembangkan website berjenis Web 2.0, namun disini akan ditunjukkan serangkaian teknologi yang dipakai, yaitu :
\begin{enumerate}
\item PHP
  Merupakan bahasa pemrograman Web yang dipergunakan paling luas di dunia web
\item CodeIgniter
  Merupakan salah satu framework PHP yang dapat digunakan untuk mengembangkan berbagai aplikasi web PHP baik yang kompleks maupun yang ringan.
\item jQuery
  Merupakan pustaka JavaScript yang akan meningkatkan web Anda dari Web 1.0 ke Web 2.0.
\end{enumerate}

Pilihan teknologi yang dipakai bisa sangat bervariasi. PHP bs digantikan oleh Java, .NET atau Ruby. CodeIgniter bs digantikan oleh CakePHP, Symphony atau Zend Framework. Dan jQuery bisa digantikan oleh Dojo Framework, scriptaculous atau MooTools. 

Selanjutnya kita akan menggali lebih dalam mengenai kombinasi teknolog yang akan dipakai pada buku ini.
\section{PHP}
\section{CodeIgniter}
\section{jQuery}